\documentclass[a4paper]{article}

\usepackage[utf8x]{inputenc}    
\usepackage[T1]{fontenc}

\usepackage[box,completemulti]{automultiplechoice}    
\begin{document}

\onecopy{10}{    

%%% beginning of the test sheet header:    

\noindent{\bf QCM  \hfill TEST}

\vspace*{.5cm}
\begin{minipage}{.4\linewidth}
\centering\large\bf Test\\ Examination on Jan., 1st, 2008\end{minipage}
\namefield{\fbox{    
                \begin{minipage}{.5\linewidth}
                  Firstname and lastname:

                  \vspace*{.5cm}\namefielddots   
                  \vspace*{1mm}
                \end{minipage}
         }}

\begin{center}\em
Duration : 10 minutes.

  No documents allowed. The use of electronic calculators is forbidden.


  Questions using the sign \multiSymbole{} may have
  zero, one or several correct answers.  Other questions have a single correct answer.

  Negative points may be attributed to \emph{very
    bad} answers.
\end{center}
\vspace{1ex}

%%% end of the header

\begin{question}{prez}    
  QUESTÃO 37 - Dado um grafo G e um vértice de origem, qual é o algoritmo de busca que descobre
todos os vértices a uma distância K do vértice origem, antes de descobrir qualquer vértice a uma
distância K+1?
  \begin{choices}
    \wrongchoice{Pré-ordem}
    \correctchoice{Largura}
    \wrongchoice{Pós-ordem}
    \wrongchoice{Profundidade}
    \wrongchoice{Simétrica}
  \end{choices}
\end{question}

\begin{questionmult}{pref}    
  QUESTÃO 38 – O programa deve ser feito de forma descendente, com a decomposição do problema
inicial em módulos, de modo a dividir as ações complexas em uma sequência de ações mais simples.
Essa técnica de programação é chamada de programação:
  \begin{choices}
    \wrongchoice{Abstrata}
    \wrongchoice{Interna}
    \wrongchoice{Declarativa}
    \wrongchoice{Sequencial}
    \correctchoice{Modular}
  \end{choices}
\end{questionmult}

% \AMCaddpagesto{3} 

}   

\end{document}
