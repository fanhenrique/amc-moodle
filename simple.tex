\documentclass[12pt,a4paper]{article}

\usepackage{csvsimple}%

\usepackage[bloc,completemulti]{automultiplechoice}

\newcommand{\Test}{
\onecopy{1}{%

%%% debut de l'en-tête des copies :  
\begin{center}
\noindent{}\fbox{\vspace*{3mm}
         \Large\bf\name{}~\forename{}\normalsize{}% 
          \vspace*{3mm}
      }
\end{center}

\noindent{\bf QCM  \hfill TEST}

\vspace*{.5cm}
\begin{minipage}{.4\linewidth}
  \centering\large\bf Test\\ Examination on Jan. 1st, 2008
\end{minipage}

\begin{center}\em
Duration : 10 minutes.

  No documents allowed. The use of electronic calculators is forbidden.

  Questions using the sign \multiSymbole{} may have
  zero, one or several correct answers.  Other questions have a single correct answer.

  Negative points may be attributed to \emph{very
    bad} answers.
\end{center}
\vspace{1ex}
%%% end of the header

\restituegroupe{general}
 

\AMCassociation{\id}

%\AMCaddpagesto{3}
    }%onecopy
}%test

%%%%§§§§§§§§§§§§§§§§§§§§§§§§§§§§§§§§§

\begin{document}
%%%Options
\AMCrandomseed{1237893}

\def\AMCformQuestion#1{{\sc Question #1 :}}

\setdefaultgroupmode{withoutreplacement}
%%% end of Options

%%% groups

\element{general}{
  \begin{question}{questao_37}    
  QUESTÃO 37 - Dado um grafo G e um vértice de origem, qual é o algoritmo de busca que descobre todos os vértices a uma distância K do vértice origem, antes de descobrir qualquer vértice a uma distância K+1?
    \begin{choices}
      \wrongchoice{Pré-ordem}
      \correctchoice{Largura}
      \wrongchoice{Pós-ordem}
      \wrongchoice{Profundidade}
      \wrongchoice{Simétrica}
    \end{choices}
  \end{question}
}

\element{general}{
  \begin{question}{questao_38}    
  QUESTÃO 38 - O programa deve ser feito de forma descendente, com a decomposição do problema inicial em módulos, de modo a dividir as ações complexas em uma sequência de ações mais simples. Essa técnica de programação é chamada de programação:
  \begin{choices}
    \wrongchoice{Abstrata}
    \wrongchoice{Interna}
    \wrongchoice{Declarativa}
    \wrongchoice{Sequencial}
    \correctchoice{Modular}
  \end{choices}
\end{question}
}

%%% end of groups

\csvreader[head to column names]{list.csv}{}{\Test}


\end{document}
